\documentclass[11pt,a4paper,english]{article}\usepackage[]{graphicx}\usepackage[]{xcolor}
% maxwidth is the original width if it is less than linewidth
% otherwise use linewidth (to make sure the graphics do not exceed the margin)
\makeatletter
\def\maxwidth{ %
  \ifdim\Gin@nat@width>\linewidth
    \linewidth
  \else
    \Gin@nat@width
  \fi
}
\makeatother

\definecolor{fgcolor}{rgb}{0.345, 0.345, 0.345}
\newcommand{\hlnum}[1]{\textcolor[rgb]{0.686,0.059,0.569}{#1}}%
\newcommand{\hlstr}[1]{\textcolor[rgb]{0.192,0.494,0.8}{#1}}%
\newcommand{\hlcom}[1]{\textcolor[rgb]{0.678,0.584,0.686}{\textit{#1}}}%
\newcommand{\hlopt}[1]{\textcolor[rgb]{0,0,0}{#1}}%
\newcommand{\hlstd}[1]{\textcolor[rgb]{0.345,0.345,0.345}{#1}}%
\newcommand{\hlkwa}[1]{\textcolor[rgb]{0.161,0.373,0.58}{\textbf{#1}}}%
\newcommand{\hlkwb}[1]{\textcolor[rgb]{0.69,0.353,0.396}{#1}}%
\newcommand{\hlkwc}[1]{\textcolor[rgb]{0.333,0.667,0.333}{#1}}%
\newcommand{\hlkwd}[1]{\textcolor[rgb]{0.737,0.353,0.396}{\textbf{#1}}}%
\let\hlipl\hlkwb

\usepackage{framed}
\makeatletter
\newenvironment{kframe}{%
 \def\at@end@of@kframe{}%
 \ifinner\ifhmode%
  \def\at@end@of@kframe{\end{minipage}}%
  \begin{minipage}{\columnwidth}%
 \fi\fi%
 \def\FrameCommand##1{\hskip\@totalleftmargin \hskip-\fboxsep
 \colorbox{shadecolor}{##1}\hskip-\fboxsep
     % There is no \\@totalrightmargin, so:
     \hskip-\linewidth \hskip-\@totalleftmargin \hskip\columnwidth}%
 \MakeFramed {\advance\hsize-\width
   \@totalleftmargin\z@ \linewidth\hsize
   \@setminipage}}%
 {\par\unskip\endMakeFramed%
 \at@end@of@kframe}
\makeatother

\definecolor{shadecolor}{rgb}{.97, .97, .97}
\definecolor{messagecolor}{rgb}{0, 0, 0}
\definecolor{warningcolor}{rgb}{1, 0, 1}
\definecolor{errorcolor}{rgb}{1, 0, 0}
\newenvironment{knitrout}{}{} % an empty environment to be redefined in TeX

\usepackage{alltt}
\usepackage[latin1]{inputenc}
\usepackage{hyperref}

\usepackage{xcolor}
\hypersetup{
    colorlinks,
    linkcolor={red!50!black},
    citecolor={blue!50!black},
    urlcolor={blue!80!black}
}

% Normal latex T1-fonts
\usepackage[T1]{fontenc}

% Figures
\usepackage{graphicx}

 % AMS-stuff
\usepackage{amsmath,amsfonts,amssymb}
\usepackage{amsbsy}

 % Misc
\usepackage{verbatim}
\usepackage{url}

\usepackage[round]{natbib}
\bibliographystyle{unsrtnat}

% Page size
\addtolength{\hoffset}{-1cm}
\addtolength{\textwidth}{2cm}
\addtolength{\voffset}{-1cm}
\addtolength{\textheight}{2cm}

% Paragraph
\setlength{\parindent}{0pt}
\setlength{\parskip}{1ex plus 0.5ex minus 0.2ex}

% Horizontal line
\newcommand{\HRule}{\rule{\linewidth}{0.5mm}}

% No page numbers
\pagestyle{empty}

% New commands:
\DeclareMathOperator{\var}{var}
\DeclareMathOperator{\cov}{cov}
\newcommand{\E}{\mathbb{E}}
\newcommand{\Prob}{\mathbb{P}}
\IfFileExists{upquote.sty}{\usepackage{upquote}}{}
\begin{document}
%\SweaveOpts{concordance=TRUE}

\begin{titlepage}

\center
\textsc{\LARGE SIRCSS-CTA}\\[1.5cm] % Name of your university/college

\HRule \\[0.4cm]
{ \huge \bfseries Lab 1: Topic models}\\[0.4cm] % Title of your document
\HRule \\[1.5cm]

\vfill

\end{titlepage}


% General Knitr options
% (this cannot be input since the file runs knitr before LaTeX)



\section{Topic Models}

This assignment uses the R package \texttt{uuml} with data and functionality to simplify coding. To install the packages just run the following:
\begin{knitrout}\small
\definecolor{shadecolor}{rgb}{0.969, 0.969, 0.969}\color{fgcolor}\begin{kframe}
\begin{alltt}
\hlkwd{install.packages}\hlstd{(}\hlstr{"remotes"}\hlstd{)}
\hlstd{remotes}\hlopt{::}\hlkwd{install_github}\hlstd{(}\hlstr{"MansMeg/IntroML"}\hlstd{,} \hlkwc{subdir} \hlstd{=} \hlstr{"rpackage"}\hlstd{)}
\hlkwd{install.packages}\hlstd{(}\hlstr{"tidytext"}\hlstd{)}
\hlkwd{install.packages}\hlstd{(}\hlstr{"topicmodels"}\hlstd{)}
\end{alltt}
\end{kframe}
\end{knitrout}

We will now analyze the classical book Pride and Prejudice by Jane Austen using a probabilistic topic model. If you have not read the book, \href{https://en.wikipedia.org/wiki/Pride_and_Prejudice}{\textbf{here}} you can read up on the story.

For this part of the assignment, \citet{griffiths2004finding} is the primary reference. I would also recommend reading \citet{blei2012probabilistic} before starting with the assignment.

We will use a Gibbs sampler to estimate ten different topics occurring in Pride and Prejudice and study where they occur. A tokenized version of the book and a \texttt{data.frame} with stopwords can be loaded as follows:

\begin{knitrout}\small
\definecolor{shadecolor}{rgb}{0.969, 0.969, 0.969}\color{fgcolor}\begin{kframe}
\begin{alltt}
\hlkwd{library}\hlstd{(uuml)}
\hlkwd{library}\hlstd{(dplyr)}
\end{alltt}


{\ttfamily\noindent\itshape\color{messagecolor}{\#\# \\\#\# Attaching package: 'dplyr'}}

{\ttfamily\noindent\itshape\color{messagecolor}{\#\# The following objects are masked from 'package:stats':\\\#\# \\\#\# \ \ \ \ filter, lag}}

{\ttfamily\noindent\itshape\color{messagecolor}{\#\# The following objects are masked from 'package:base':\\\#\# \\\#\# \ \ \ \ intersect, setdiff, setequal, union}}\begin{alltt}
\hlkwd{library}\hlstd{(tidytext)}
\hlkwd{data}\hlstd{(}\hlstr{"pride_and_prejudice"}\hlstd{)}
\hlkwd{data}\hlstd{(}\hlstr{"stopwords"}\hlstd{)}
\end{alltt}
\end{kframe}
\end{knitrout}


\begin{enumerate}
\item As a first step, we will remove stopwords (common English words without much semantic information):
\begin{knitrout}\small
\definecolor{shadecolor}{rgb}{0.969, 0.969, 0.969}\color{fgcolor}\begin{kframe}
\begin{alltt}
\hlstd{pap} \hlkwb{<-} \hlstd{pride_and_prejudice}
\hlstd{pap} \hlkwb{<-} \hlkwd{anti_join}\hlstd{(pap,} \hlkwc{y} \hlstd{= stopwords[stopwords}\hlopt{$}\hlstd{lexicon} \hlopt{==} \hlstr{"snowball"}\hlstd{,])}
\end{alltt}


{\ttfamily\noindent\itshape\color{messagecolor}{\#\# Joining with `by = join\_by(word)`}}\end{kframe}
\end{knitrout}

\item Then we will remove rare words. Here we remove words that occur less than five times.
\begin{knitrout}\small
\definecolor{shadecolor}{rgb}{0.969, 0.969, 0.969}\color{fgcolor}\begin{kframe}
\begin{alltt}
\hlstd{word_freq} \hlkwb{<-} \hlkwd{table}\hlstd{(pap}\hlopt{$}\hlstd{word)}
\hlstd{rare_words} \hlkwb{<-} \hlkwd{data.frame}\hlstd{(}\hlkwc{word} \hlstd{=} \hlkwd{names}\hlstd{(word_freq[word_freq} \hlopt{<=} \hlnum{5}\hlstd{]),} \hlkwc{stringsAsFactors} \hlstd{=} \hlnum{FALSE}\hlstd{)}
\hlstd{pap} \hlkwb{<-} \hlkwd{anti_join}\hlstd{(pap,} \hlkwc{y} \hlstd{= rare_words)}
\end{alltt}


{\ttfamily\noindent\itshape\color{messagecolor}{\#\# Joining with `by = join\_by(word)`}}\end{kframe}
\end{knitrout}
\item Now we have a corpus we can used to implement a probabilistic topic model. We do this by using the \texttt{topicmodels} R package. As a first step we will compute a document term matrix using the \texttt{tm} package, where we treat each paragraph as a document. How many documents and terms (word types) do you have?
\begin{knitrout}\small
\definecolor{shadecolor}{rgb}{0.969, 0.969, 0.969}\color{fgcolor}\begin{kframe}
\begin{alltt}
\hlkwd{library}\hlstd{(tm)}
\hlstd{crp} \hlkwb{<-} \hlkwd{aggregate}\hlstd{(pap}\hlopt{$}\hlstd{word,} \hlkwc{by} \hlstd{=} \hlkwd{list}\hlstd{(pap}\hlopt{$}\hlstd{paragraph),} \hlkwc{FUN} \hlstd{= paste0,} \hlkwc{collapse} \hlstd{=} \hlstr{" "}\hlstd{)}
\hlkwd{names}\hlstd{(crp)} \hlkwb{<-} \hlkwd{c}\hlstd{(}\hlstr{"paragraph"}\hlstd{,} \hlstr{"text"}\hlstd{)}
\hlstd{s} \hlkwb{<-} \hlkwd{SimpleCorpus}\hlstd{(}\hlkwd{VectorSource}\hlstd{(crp}\hlopt{$}\hlstd{text))}
\hlstd{m} \hlkwb{<-} \hlkwd{DocumentTermMatrix}\hlstd{(s)}
\end{alltt}
\end{kframe}
\end{knitrout}
\item To compute a topic model with ten topics, we use a Gibbs sampling algorithm. Below is an example of how we can run a Gibbs sampler for 2000 iterations. Run your topic model for 2000 iterations.
\begin{knitrout}\small
\definecolor{shadecolor}{rgb}{0.969, 0.969, 0.969}\color{fgcolor}\begin{kframe}
\begin{alltt}
\hlkwd{library}\hlstd{(topicmodels)}
\hlstd{K} \hlkwb{<-} \hlnum{10}
\hlcom{# Note: delta is beta in Griffith and Steyvers (2004) notation.}
\hlstd{control} \hlkwb{<-} \hlkwd{list}\hlstd{(}\hlkwc{keep} \hlstd{=} \hlnum{1}\hlstd{,} \hlkwc{delta} \hlstd{=} \hlnum{0.1}\hlstd{,} \hlkwc{alpha} \hlstd{=} \hlnum{1}\hlstd{,} \hlkwc{iter} \hlstd{=} \hlnum{2000}\hlstd{)}
\hlstd{tm} \hlkwb{<-} \hlkwd{LDA}\hlstd{(m,} \hlkwc{k} \hlstd{= K,} \hlkwc{method} \hlstd{=} \hlstr{"Gibbs"}\hlstd{, control)}
\end{alltt}
\end{kframe}
\end{knitrout}
\item In the \texttt{uuml} R package you have three convenience functions to extract $\Theta$, $\Phi$ and the log-likelihood values at each iteration. This is the parameter notation used in \citet{griffiths2004finding}.
\begin{knitrout}\small
\definecolor{shadecolor}{rgb}{0.969, 0.969, 0.969}\color{fgcolor}\begin{kframe}
\begin{alltt}
\hlkwd{library}\hlstd{(uuml)}
\hlstd{lls} \hlkwb{<-} \hlkwd{extract_log_liks}\hlstd{(tm)}
\hlstd{theta} \hlkwb{<-} \hlkwd{extract_theta}\hlstd{(tm)}
\hlstd{phi} \hlkwb{<-} \hlkwd{extract_phi}\hlstd{(tm)}
\end{alltt}
\end{kframe}
\end{knitrout}
\item As a first step, check that the model has converged by visualizing the log-likelihood over epochs/iterations. Does it seem like the model have converged?
\item Extract the 20 top words for each topic (i.e. the words with the highest probability in each topic). Choose two topics you find coherent/best (the top words seem to belong together). Interpret these two topics based on the storyline of the book. What have these two topics captured?
\item Visualize these two topics evolve over the paragraphs in the books by plotting the $\theta$ parameters for that topic over time (paragraphs) in the book. Think of this as the time-line of the book. On the y-axis, you should plot $\theta_i$ for your chosen topic $i$ and the x-axis should be the paragraph number (first paragraph has number 1 and so forth).
\item How do these two chosen topics evolve over the course in the book? If you want, you can take a rolling mean of the theta parameters to more easily show the changes in the topic over the book. \emph{Hint!} Here \texttt{zoo::rollmean()} might be a good function to use.
\item Test to change the number of topics and do your own analysis of the novel when you feel you have a good number of topics.

\end{enumerate}

\bibliography{bibliography}

\end{document}

%%% Local Variables:
%%% mode: latex
%%% TeX-master: t
%%% End:
